\documentclass[10pt,a4paper]{article}

\usepackage[spanish,activeacute,es-tabla]{babel}
\usepackage[utf8]{inputenc}
\usepackage{ifthen}
\usepackage{listings}
\usepackage{dsfont}
\usepackage{subcaption}
\usepackage{amsmath}
\usepackage[strict]{changepage}
\usepackage[top=1cm,bottom=2cm,left=1cm,right=1cm]{geometry}%
\usepackage{color}%
\newcommand{\tocarEspacios}{%
	\addtolength{\leftskip}{1.5em}%
	\setlength{\parindent}{0em}%
}

% Especificacion de procs

\newcommand{\In}{\textsf{in }}
\newcommand{\Out}{\textsf{out }}
\newcommand{\Inout}{\textsf{inout }}

\newcommand{\encabezadoDeProc}[4]{%
	% Ponemos la palabrita problema en tt
	%  \noindent%
	{\normalfont\bfseries\ttfamily proc}%
	% Ponemos el nombre del problema
	\ %
	{\normalfont\ttfamily #2}%
	\
	% Ponemos los parametros
	(#3)%
	\ifthenelse{\equal{#4}{}}{}{%
		% Por ultimo, va el tipo del resultado
		\ : #4}
}

\newenvironment{proc}[4][res]{%
	
	% El parametro 1 (opcional) es el nombre del resultado
	% El parametro 2 es el nombre del problema
	% El parametro 3 son los parametros
	% El parametro 4 es el tipo del resultado
	% Preambulo del ambiente problema
	% Tenemos que definir los comandos requiere, asegura, modifica y aux
	\newcommand{\requiere}[2][]{%
		{\normalfont\bfseries\ttfamily requiere}%
		\ifthenelse{\equal{##1}{}}{}{\ {\normalfont\ttfamily ##1} :}\ %
		\{\ensuremath{##2}\}%
		{\normalfont\bfseries\,\par}%
	}
	\newcommand{\asegura}[2][]{%
		{\normalfont\bfseries\ttfamily asegura}%
		\ifthenelse{\equal{##1}{}}{}{\ {\normalfont\ttfamily ##1} :}\
		\{\ensuremath{##2}\}%
		{\normalfont\bfseries\,\par}%
	}
	\renewcommand{\aux}[4]{%
		{\normalfont\bfseries\ttfamily aux\ }%
		{\normalfont\ttfamily ##1}%
		\ifthenelse{\equal{##2}{}}{}{\ (##2)}\ : ##3\, = \ensuremath{##4}%
		{\normalfont\bfseries\,;\par}%
	}
	\renewcommand{\pred}[3]{%
		{\normalfont\bfseries\ttfamily pred }%
		{\normalfont\ttfamily ##1}%
		\ifthenelse{\equal{##2}{}}{}{\ (##2) }%
		\{%
		\begin{adjustwidth}{+3em}{}
			\ensuremath{##3}
		\end{adjustwidth}
		\}%
		{\normalfont\bfseries\,\par}%
	}
	
	\newcommand{\res}{#1}
	\vspace{1ex}
	\noindent
	\encabezadoDeProc{#1}{#2}{#3}{#4}
	% Abrimos la llave
	\par%
	\tocarEspacios
}
{
	% Cerramos la llave
	\vspace{1ex}
}

\newcommand{\aux}[4]{%
	{\normalfont\bfseries\ttfamily\noindent aux\ }%
	{\normalfont\ttfamily #1}%
	\ifthenelse{\equal{#2}{}}{}{\ (#2)}\ : #3\, = \ensuremath{#4}%
	{\normalfont\bfseries\,;\par}%
}

\newcommand{\pred}[3]{%
	{\normalfont\bfseries\ttfamily\noindent pred }%
	{\normalfont\ttfamily #1}%
	\ifthenelse{\equal{#2}{}}{}{\ (#2) }%
	\{%
	\begin{adjustwidth}{+1em}{}
		\ensuremath{#3}
	\end{adjustwidth}
	\}%
	{\normalfont\bfseries\,\par}%
}

% Tipos

\newcommand{\nat}{\ensuremath{\mathds{N}}}
\newcommand{\ent}{\ensuremath{\mathds{Z}}}
\newcommand{\float}{\ensuremath{\mathds{R}}}
\newcommand{\bool}{\ensuremath{\mathsf{Bool}}}
\newcommand{\cha}{\ensuremath{\mathsf{Char}}}
\newcommand{\str}{\ensuremath{\mathsf{String}}}

% Logica

\newcommand{\True}{\ensuremath{\mathrm{true}}}
\newcommand{\False}{\ensuremath{\mathrm{false}}}
\newcommand{\Then}{\ensuremath{\rightarrow}}
\newcommand{\Iff}{\ensuremath{\leftrightarrow}}
\newcommand{\implica}{\ensuremath{\longrightarrow}}
\newcommand{\IfThenElse}[3]{\ensuremath{\mathsf{if}\ #1\ \mathsf{then}\ #2\ \mathsf{else}\ #3\ \mathsf{fi}}}
\newcommand{\yLuego}{\land _L}
\newcommand{\oLuego}{\lor _L}
\newcommand{\implicaLuego}{\implica _L}

\newcommand{\cuantificador}[5]{%
	\ensuremath{(#2 #3: #4)\ (%
		\ifthenelse{\equal{#1}{unalinea}}{
			#5
		}{
			$ % exiting math mode
			\begin{adjustwidth}{+2em}{}
				$#5$%
			\end{adjustwidth}%
			$ % entering math mode
		}
		)}
}

\newcommand{\existe}[4][]{%
	\cuantificador{#1}{\exists}{#2}{#3}{#4}
}
\newcommand{\paraTodo}[4][]{%
	\cuantificador{#1}{\forall}{#2}{#3}{#4}
}

%listas

\newcommand{\TLista}[1]{\ensuremath{seq \langle #1\rangle}}
\newcommand{\lvacia}{\ensuremath{[\ ]}}
\newcommand{\lv}{\ensuremath{[\ ]}}
\newcommand{\longitud}[1]{\ensuremath{|#1|}}
\newcommand{\cons}[1]{\ensuremath{\mathsf{addFirst}}(#1)}
\newcommand{\indice}[1]{\ensuremath{\mathsf{indice}}(#1)}
\newcommand{\conc}[1]{\ensuremath{\mathsf{concat}}(#1)}
\newcommand{\cab}[1]{\ensuremath{\mathsf{head}}(#1)}
\newcommand{\cola}[1]{\ensuremath{\mathsf{tail}}(#1)}
\newcommand{\sub}[1]{\ensuremath{\mathsf{subseq}}(#1)}
\newcommand{\en}[1]{\ensuremath{\mathsf{en}}(#1)}
\newcommand{\cuenta}[2]{\mathsf{cuenta}\ensuremath{(#1, #2)}}
\newcommand{\suma}[1]{\mathsf{suma}(#1)}
\newcommand{\twodots}{\ensuremath{\mathrm{..}}}
\newcommand{\masmas}{\ensuremath{++}}
\newcommand{\matriz}[1]{\TLista{\TLista{#1}}}
\newcommand{\seqchar}{\TLista{\cha}}

\renewcommand{\lstlistingname}{Código}
\lstset{% general command to set parameter(s)
	language=Java,
	morekeywords={endif, endwhile, skip},
	basewidth={0.47em,0.40em},
	columns=fixed, fontadjust, resetmargins, xrightmargin=5pt, xleftmargin=15pt,
	flexiblecolumns=false, tabsize=4, breaklines, breakatwhitespace=false, extendedchars=true,
	numbers=left, numberstyle=\tiny, stepnumber=1, numbersep=9pt,
	frame=l, framesep=3pt,
	captionpos=b,
}

\usepackage{caratula} % Version modificada para usar las macros de algo1 de ~> https://github.com/bcardiff/dc-tex

\newcommand{\predRef}{\texttt}
\newcommand{\var}{\text}
\renewcommand{\wp}{\text{wp}}
\newcommand{\isDef}{\text{def}}

\titulo{Trabajo práctico 1: Especificación y WP}
\subtitulo{Elecciones nacionales}

\fecha{\today}

\materia{Algoritmos y Estructuras de Datos}
\grupo{Grupo 42}

\integrante{Bossi, Tomás}{50/17}{tomasbossi97@gmail.com}
\integrante{Dominguez, Rocio Julieta}{798/22}{rociodominguezcpm@gmail.com}
\integrante{Stabile, Delfina}{819/22}{delfistabile18@gmail.com}
\integrante{Ziger, Bruno Martín}{218/23}{ziger.bruno@gmail.com}
% Pongan cuantos integrantes quieran

% Declaramos donde van a estar las figuras
% No es obligatorio, pero suele ser comodo
\graphicspath{{../static/}}

\begin{document}

\maketitle

\section{Especificación}

\subsection*{Aclaraciones sobre las especificaciones}
\begin{itemize}
  \item En todos los puntos, asumimos que no pueden haber empates entre partidos, ni tampoco entre cualquiera de los partidos y los votos en blanco, para todos los escrutinios. Esto lo implementamos con el predicado sinRepetidos, en los casos donde el funcionamiento del programa depende de esta condición.
  \item En la especificacion del problema hayBallotage, excluimos en los predicados la ultima posición mediante la función $\predRef{subseq}$ para excluir los votos en blancos de los máximos (\ref{ejercicio_1}).
  \item Decidimos aplicar el umbral del $3\%$ en el ejercicio 4 de la siguiente manera: para aquellos partidos cuyo porcentaje de votos no supere el umbral del $3\%$, su fila en la matriz D'Hondt será una lista de ceros (''aplanando'' los resultados pequeños) (\ref{fila_ceros}). Entonces el caso donde los coeficientes son $0$ se excluye del requiere de que la matriz tenga todos los cocientes distintos. Con esto, somos consistentes y evitamos el problema de los votos en blanco en el ejercicio 5 (\ref{ejercicio_5}).
\end{itemize}

\subsection{hayBallotage} \label{ejercicio_1}
\begin{proc}{hayBallotage}{\In escrutinio : \TLista{\ent}}{\bool}
	% \modifica{parametro1, parametro2, ...}
	\requiere{\longitud{\var{escrutinio}} \geq 3 \land \predRef{sinRepetidos}(\var{escrutinio}) \land \predRef{sinNegativos}(\var{escrutinio})}
    \asegura{\var{res} = \False \Iff \predRef{elMaximoTieneMasDeCuarentaYCinco}(\var{escrutinio}) \lor \\
    (\predRef{elMaximoTieneMasDeCuarenta}(\var{escrutinio}) \land
    \predRef{elMaximoTieneDiferenciaMasDeDiez}(\var{escrutinio}))}
    
    \aux{suma}{\In $l$ : \TLista{\ent}}{\ent}{\sum\limits_{i=0}^{|l|-1} l[i]}
    \aux{porcentajeDelTotal}{\In $l$ : \TLista{\ent}, \In $i$ : \ent}{\ent}{100 \frac{l[i]}{ \predRef{suma}(l)}}

    \pred{sinNegativos}{$l$: \TLista{\ent}}{
        \paraTodo[unalinea]{i}{\ent}{0\leq i < \longitud{l}\implicaLuego l[i]\geq 0
    }}
    \pred{sinRepetidos}{$l$: \TLista{\ent}}{
        \paraTodo[unalinea]{i, j}{\ent}{0\leq i < j < \longitud{l}\implicaLuego l[i]\neq l[j]}
    }
	\pred{esIdDelMaximo}{$l$ : \TLista{\ent}, $i$ : \ent} {
        0 \leq i < \longitud{l} \yLuego
        \paraTodo[unalinea]{j}{\ent}{0\leq j < \longitud{l}\implicaLuego l[j]\leq l[i]}
    }
	\pred{esIdDelSegundoMaximo}{$l$ : \TLista{\ent}, $i$ : \ent} {
        0 \leq i < \longitud{l} \yLuego
        (\neg \predRef{esIdDelMaximo}(l, i) \land
        \paraTodo[unalinea]{j}{\ent}{0\leq j < \longitud{l} \land \neg \predRef{esIdDelMaximo}(l, j)\implicaLuego l[j]\leq l[i])}
    }
    \pred{elMaximoTieneMasDeCuarentaYCinco}{$l$ : \TLista{\ent}} {
        \paraTodo[unalinea]{i}{\ent}{\predRef{esIdDelMaximo}(\predRef{subseq}(l, 0, \longitud{l} - 1), i)\implicaLuego \predRef{porcentajeDelTotal}(l, i) > 45}
    }
    \pred{elMaximoTieneMasDeCuarenta}{$l$ : \TLista{\ent}} {
        \paraTodo[unalinea]{i}{\ent}{\predRef{esIdDelMaximo}(\predRef{subseq}(l, 0, \longitud{l} - 1), i)\implicaLuego \predRef{porcentajeDelTotal}(l, i) > 40}
    }
    \pred{elMaximoTieneDiferenciaMasDeDiez}{$l$ : \TLista{\ent}} {
        \paraTodo{i, j}{\ent}{\predRef{esIdDelMaximo}(\predRef{subseq}(l, 0, \longitud{l} - 1), i)\land \predRef{esIdDelSegundoMaximo}(\predRef{subseq}(l, 0, \longitud{l} - 1), j) \implicaLuego\\ \predRef{porcentajeDelTotal}(l, i) > \predRef{porcentajeDelTotal}(l, j) + 10}
    }
\end{proc}

\subsection{hayFraude}
\begin{proc}{hayFraude}{\In escrutinio\_presidencial : \TLista{\ent}, \In escrutinio\_senadores : \TLista{\ent}, \In escrutinio\_diputados : \TLista{\ent}}{\bool}
    \requiere{\longitud{\var{escrutinio\_presidencial}} = \longitud{\var{escrutinio\_senadores}} = \longitud{\var{escrutinio\_diputados}} > 0}

    \asegura{\var{res} = \False \Iff (\predRef{suma}(\var{escrutinio\_presidencial}) = \predRef{suma}(\var{escrutinio\_senadores}) = \predRef{suma}(\var{escrutinio\_diputados}))}

    \aux{suma}{\In $l$ : \TLista{\ent}}{\ent}{\sum\limits_{i=0}^{|l|-1} l[i]}
\end{proc}

\subsection{obtenerSenadoresEnProvincia}
\begin{proc}{obtenerSenadoresEnProvincia}{\In escrutinio : \TLista{\ent}}{\ent$\times$\ent}
	\requiere{\longitud{\var{escrutinio}} \geq 3 \land \predRef{sinRepetidos}(\var{escrutinio}) \land \predRef{sinNegativos}(\var{escrutinio})}

    \asegura{\predRef{esIdDelMaximo}(\predRef{subseq}(\var{escrutinio}, 0, \longitud{\var{escrutinio}} - 1), res_0)}
    \asegura{\predRef{esIdDelSegundoMaximo}(\predRef{subseq}(\var{escrutinio}, 0, \longitud{\var{escrutinio}} - 1), res_1)}

    \pred{sinRepetidos}{$l$ : \TLista{\ent}}{
        \paraTodo[unalinea]{i, j}{\ent}{0\leq i < j < \longitud{l}\implicaLuego l[i]\neq l[j]}
    }
    \pred{sinNegativos}{$l$: \TLista{\ent}}{
        \paraTodo[unalinea]{i}{\ent}{0\leq i < \longitud{l}\implicaLuego l[i]\geq 0
    }}
	\pred{esIdDelMaximo}{$l$ : \TLista{\ent}, $i$ : \ent} {
        0 \leq i < \longitud{l} \yLuego
        \paraTodo[unalinea]{j}{\ent}{0\leq j < \longitud{l} \implicaLuego l[j]\leq l[i]}
    }
	\pred{esIdDelSegundoMaximo}{$l$ : \TLista{\ent}, $i$ : \ent} {
        0 \leq i < \longitud{l} \land
        \neg \predRef{esIdDelMaximo}(l, i) \land
        \paraTodo[unalinea]{j}{\ent}{(0\leq j < \longitud{l} \land \neg \predRef{esIdDelMaximo}(l, j))\implicaLuego l[j]\leq l[i]}
    }
\end{proc}

\subsection{calcularDHondtEnProvincia} 
\begin{proc}{calcularDHondtEnProvincia}{\In cant\_bancas : \ent, \In escrutinio : \TLista{\ent}}{ \TLista{\TLista{\ent}}}
    \requiere{\longitud{\var{escrutinio}}\geq 2\land \predRef{sinNegativos}(\var{escrutinio})}
    \requiere{
        \paraTodo[unalinea]{i, j, k, l}{\ent}{0\leq i, k < \longitud{\var{escrutinio}} - 1\land 0\leq j, l < \var{cant\_bancas}\implicaLuego (i = k\land j = l)\lor \frac{\var{escrutinio}[i]}{j+1}\neq \frac{\var{escrutinio}[k]}{l+1}}
    }
    
    \asegura{\longitud{\var{res}} = \longitud{\var{escrutinio}} - 1}
    \asegura{\paraTodo[unalinea]{i, j}{\ent}{0\leq i < \longitud{\var{escrutinio} - 1}\implicaLuego \longitud{\var{res}[i]} = \var{cant\_bancas}}}
    \asegura{\paraTodo[unalinea]{i, j, k, l}{\ent}{0\leq i, k < \longitud{\var{escrutinio}} - 1\land 0\leq j, l < \var{cant\_bancas}\implicaLuego (i = k\land j = l)\lor \var{res}[i][j] = 0\lor \var{res}[i][j]\neq \var{res}[k][l]}}
    \asegura{\paraTodo[unalinea]{i, j}{\ent}{0 \leq i < \longitud{\var{escrutinio}} - 1\land 0 \leq j < \var{cant\_bancas}\implicaLuego \var{res}[i][j] = \predRef{DHont}(\var{escrutinio}, \var{escrutinio}[i], j)}}
    
    \aux{DHondt}{\In escrutinio : \TLista{\ent}, \In votos : \ent, \In cociente : \ent}{\float}{\IfThenElse{100\frac{\var{votos}}{\predRef{suma}(\var{escrutinio})}>3}{\frac{\var{votos}}{\var{cociente} + 1}}{0}}
    \label{fila_ceros}

    \pred{sinNegativos}{$l$ : \TLista{\ent}} {
        \paraTodo[unalinea]{i}{\ent}{0\leq i < \longitud{l}\implicaLuego l[i]\geq 0
    }}
\end{proc}

\subsection{obtenerDiputadosEnProvincia}\label{ejercicio_5}
\begin{proc}{obtenerDiputadosEnProvincia}{\In cant\_bancas : \ent, \In escrutinio : \TLista{\ent}, \In dHondt : \TLista{\TLista{\ent}}}{\TLista{\ent}}
    \requiere{\longitud{\var{dHondt}} > 0\land \var{cant\_bancas} > 0\land \predRef{sinNegativos}(\var{dHondt})\land \predRef{noTodosCeros}(\var{dHondt})}
    \requiere{\paraTodo[unalinea]{i, j, k, l}{\ent}{0\leq i, k < \longitud{\var{dHondt}}\land 0\leq j, l < \var{cant\_bancas}\implicaLuego (i = k\land j = l)\lor \var{dHondt}[i][j] = 0\lor \var{dHondt}[i][j]\neq \var{dHondt}[k][l]}}
    
    \asegura{\longitud{\var{res}} = \longitud{\var{dHondt}}}
    \asegura{\predRef{suma}(\var{res}) = \var{cant\_bancas}}
    \asegura{\paraTodo[unalinea]{i}{\ent}{\var{res}[i] = \predRef{cuantosSuperan}(\var{dHondt}, i, \var{cant\_bancas})}}
    %\asegura{(\paraTodo[unalinea]{i}{\ent}{0\leq j < \longitud{res} \implicaLuego res[j] = \sum\limits_{i=0}^{|dHondt|-1} $If esIdDelMaximo(dHondt[i],j) then 1 else 0 fi$)}}
    \aux{suma}{\In $l$ : \TLista{\ent}}{\ent}{\sum\limits_{i=0}^{|l|-1} l[i]}
    \aux{posicion}{\In $M$ : \matriz{\ent}, \In $t$ : \ent}{\ent}{\sum\limits_{i=0}^{\longitud{M}-1}\sum\limits_{j=0}^{\longitud{M[i]}-1} \IfThenElse{M[i][j]\geq t}{1}{0} }
    \aux{cuantosSuperan}{\In $M$ : \matriz{\ent}, \In $i$ : \ent, \In cant\_bancas : \ent}{\ent}
    {\\\sum\limits_{j=0}^{\longitud{M[i]}-1}{\IfThenElse{\predRef{posicion}(M, M[i][j])\leq \var{cant\_bancas}}{1}{0}}}
    
    \pred{sinNegativos}{$M$ : \matriz{\ent}} {
        \paraTodo[unalinea]{i, j}{\ent}{(0\leq i < \longitud{M}\yLuego 0\leq j < \longitud{M[i]})\implicaLuego M[i][j] \geq 0}
    }
    \pred{noTodosCeros}{$M$ : \matriz{\ent}} {
        \existe[unalinea]{i, j}{\ent}{0\leq i < \longitud{M}\yLuego 0\leq j < \longitud{M[i]}\yLuego M[i][j] > 0}
    }
\end{proc}

\subsection{validarListasDiputadosEnProvincia}
\begin{proc}{validarListasDiputadosEnProvincia}{\In cant\_bancas : \ent, \In listas : \matriz{dni : \ent\times genero : \ent}}{\bool}
    \requiere{\var{cant\_bancas} > 0\land \longitud{\var{listas}} > 0}
    \requiere{\paraTodo[unalinea]{i}{\ent}{0\leq i < \longitud{\var{listas}}\implicaLuego (\predRef{tieneDnisValidos}(\var{listas}[i])\land \predRef{tieneGenerosValidos}(\var{listas}[i])}}
    \asegura{\var{res} = \True \Iff \paraTodo[unalinea]{i}{\ent}{0\leq i < \longitud{\var{listas}}\implicaLuego \longitud{\var{listas}[i]} = \var{cant\_bancas} \land \predRef{alternaGeneros}(\var{lista}[i])}}
    \pred{tieneDnisValidos}{lista : \TLista{\ent\times \ent}} {
        \paraTodo[unalinea]{i}{\ent}{0\leq i < \longitud{\var{lista}}\implicaLuego (\var{lista}[i]_0 > 0)}
    }
    \pred{tieneGenerosValidos}{lista : \TLista{\ent\times \ent}} {
        \paraTodo[unalinea]{i}{\ent}{0\leq i < \longitud{\var{lista}}\implicaLuego (\var{lista}[i]_1 = 1\lor \var{lista}[i]_1 = 2)}
    }
    \pred{alternaGeneros}{lista : \TLista{dni : \ent\times genero : \ent}} {
        \paraTodo[unalinea]{i, j}{\ent}{0\leq i, j < \longitud{\var{lista}}\implicaLuego {\var{lista}[i]_2 = \var{lista}[j]_2 \Iff i \% 2 = j \% 2}}
    }
\end{proc}

\section{Implementaciones}

\subsection{hayBallotage}
\begin{minipage}[t]{\textwidth}
    \begin{lstlisting}[label=code:for]
i := 0;
total_votos := 0;
maximo_sin_blanco := 0;
segundo_sin_blanco := 0;

while (i < |escrutinio|) do
    suma := suma + escrutinio[i]
    if (i < |escrutinio| - 1 && escrutinio[i] > maximo_sin_blanco) then
        segundo_sin_blanco := maximo_sin_blanco
        maximo_sin_blanco := escrutinio[i]
    else
        if (escrutinio[i] > segundo_sin_blanco) then
            segundo_sin_blanco := escrutinio[i]
        else
            skip
        endif;
    endif;
    i := i + 1
endwhile;

porcentaje_maximo := 100 * maximo_sin_blanco / total_votos;
porcentaje_segundo := 100 * segundo_sin_blanco / total_votos;

if (porcentaje_maximo > 45 || (porcentaje_maximo > 40 && porcentaje_maximo - porcentaje_segundo > 10)) then
	res := false
else
    res := true
endif;
    \end{lstlisting}
\end{minipage}

\subsection{hayFraude}
\begin{minipage}[t]{\textwidth}
    \begin{lstlisting}[label=code:for]
i := 0;
suma_presidencial := 0;
suma_senadores := 0;
suma_diputados := 0;

while (i < |escrutinio_presidencial|) do
    suma_presidencial := suma_presidencial + escrutinio_presidencial[i]
    suma_senadores := suma_senadores + escrutinio_senadores[i]
    suma_diputados := suma_diputados + escrutinio_diputados[i]
    i := i + 1
endwhile;

if (suma_presidencial == suma_senadores && suma_senadores == suma_diputados) then
	res := false
else
    res := true
endif;
    \end{lstlisting}
\end{minipage}

\subsection{obtenerSenadoresEnProvincia}
\begin{minipage}[t]{\textwidth}
    \begin{lstlisting}[label=code:for]
if (escrutinio[0] > escrutinio[1]) then
    indice_maximo = 0
    indice_segundo = 1
else
    indice_maximo = 1
    indice_segundo = 0
endif;
i := 2;
while (i < |escrutinio| - 1) do
    if (escrutinio[i] > escrutinio[indice_maximo]) then
        indice_segundo := indice_maximo
        indice_maximo := i
    else
        if (escrutinio[i] > escrutinio[indice_segundo]) then
            indice_segundo := i
        else
            skip
        endif;
    endif;
    i := i + 1
endwhile;

res := (indice_maximo, indice_segundo);
    \end{lstlisting}
\end{minipage}
% i := 0;
% indice_maximo = -1;
% indice_segundo = -1;

% while (i < |escrutinio| - 1) do
%     if (indice_maximo == -1 || escrutinio[i] > escrutinio[indice_maximo]) then
%         indice_segundo := indice_maximo
%         indice_maximo := i
%     else
%         if (indice_segundo == -1 || escrutinio[i] > escrutinio[indice_segundo]) then
%             indice_segundo := escrutinio[i]
%         else
%             skip
%         endif;
%     endif;
%     i := i + 1
% endwhile;

% res := (indice_maximo, indice_segundo);

% res := setAt(res, 0, indice_maximo);
% res := setAt(res, 1, indice_segundo);

\subsection{validarListasDiputadosEnProvincia}
\begin{minipage}[t]{\textwidth}
    \begin{lstlisting}[label=code:for]
res := true;
i := 0;

while (i < |listas|) do
    lista := listas[i]
    
    if (|lista| != cant_bancas) then
        res := false
    else
        skip
    endif;
    
    j := 0
    while (j < |lista| - 1) do
        if (lista[j][1] == lista[j+1][1]) then
            res := false
        else
            skip
        endif;
        j := j + 1;
    endwhile;
    
    i := i + 1;
endwhile;
    \end{lstlisting}
\end{minipage}

\section{Demostraciones formales de correctitud}
\subsection*{Aclaraciones sobre las demostraciones}
\begin{itemize}
    \item Usamos la notación $Q_{a_1, a_2, \ldots, a_k}^{b_1, b_2, \ldots, b_k}$ indicando que estamos reemplazando muchas variables libres de $Q$ secuencialmente, desde $a_i := b_1$ hasta $a_k := b_k$.
\end{itemize}
\subsection{hayFraude}
Nombremos los predicados que vamos a utilizar para el método de la weakest precondition: 
\begin{itemize}
    \item El auxiliar para sumar los elementos de una lista: $\predRef{suma}(l) = \sum_{i=0}^{|l|-1}{l[i]}$
    \item Precondición: $P = \{\longitud{\var{escrutinio\_presidencial}} = \longitud{\var{escrutinio\_senadores}} = \longitud{\var{escrutinio\_diputados}} > 0\}$
    \item Postcondición: $Q = \{\var{res} = \False \Iff A\}$, donde $A = \{\predRef{suma}(\var{escrutinio\_presidencial}) = \predRef{suma}(\var{escrutinio\_senadores}) = \predRef{suma}(\var{escrutinio\_diputados})\}$
    \item Condición del ciclo: $B = \{ i < |\var{escrutinio\_presidencial}| \}$
    \item Condición del if: $C = \{\var{suma\_presidencial} = \var{suma\_senadores} = \var{suma\_diputados}\}$
    \item Invariantes: $I_1 = \{ 0\leq i\leq |\var{escrutinio\_presidencial}|\yLuego \var{suma\_presidencial} = \sum_{k=0}^{i-1}{\var{escrutinio\_presidencial}[k]} \}$
    \item $I_2 = \{ 0\leq i\leq |\var{escrutinio\_diputados}|\yLuego \var{suma\_diputados} = \sum_{k=0}^{i-1}{\var{escrutinio\_diputados}[k]} \}$
    \item $I_3 = \{ 0\leq i\leq |\var{escrutinio\_senadores}|\yLuego \var{suma\_senadores} = \sum_{k=0}^{i-1}{\var{escrutinio\_senadores}[k]} \}$
\end{itemize}
Para demostrar la correctitud del ciclo, proponemos:
\begin{itemize}
    \item $P_c = \{i = 0\land \var{suma\_presidencial} = 0\land \var{suma\_diputados} = 0\land \var{suma\_senadores} = 0\land P\}$\\
    Por ser razonable que $P_c$ sea la conjunción entre $P$ y el estado resultante de las instrucciones previas al ciclo.
    \item $I = \{P \land I_1 \land I_2 \land I_3\}$\\
    Porque refleja lo que desde la intuición comprendemos que debe ocurrir al principio, durante (luego de cada iteración) y al final de la ejecución del ciclo.
    \item $Q_c = \{\var{suma\_presidencial} = \predRef{suma}(\var{escrutinio\_presidencial}) \land \var{suma\_senadores} = \predRef{suma}(\var{escrutinio\_senadores}) \land \var{suma\_diputados} = \predRef{suma}(\var{escrutinio\_diputados})\}$.
    
    Porque esto garantiza que las variables $\var{suma\_presidencial}$, $\var{suma\_senadores}$ y $\var{suma\_diputados}$ contienen la información que deberían.
    \item $f_v = |\var{escrutinio\_presidencial}| - i$\\
    Porque fv decrece en cada iteración del ciclo y se hace 0 luego de la última, cuando $i = |\var{escrutinio\_presicencial}|$.
\end{itemize}

Primero vamos a demostrar la correctitud del if que le sigue al ciclo, al cual llamamos $S_{\text{if}}$. Para ello, calculamos la wp de $S_{\text{if}}$ con la postcondición $Q$. Recordar que llamamos $C$ a la condición de este if.

Aplicando los axiomas vistos en clase sobre la wp:
$$\wp (\IfThenElse{C}{\var{res} := \False}{\var{res} := \True}, Q) \equiv \isDef(C) \yLuego ((C \land \wp(res := \False, Q)) \lor (\neg C \land \wp(res := \True, Q)))$$

Y repetimos lo mismo en las wp de cada asignación por separado:
$$\wp (\var{res} := \False, Q) \equiv \isDef(\False) \land Q_{\var{\False}}^{\var{res}}\equiv \{\False = \False \Iff A\} \equiv A$$
$$\wp (\var{res} := \True, Q) \equiv \isDef(\True) \land Q_{\var{\True}}^{\var{res}}\equiv \{\True = \False \Iff A\} \equiv \neg A$$

Combinando:
$$\wp(\IfThenElse{C}{\var{res} := \False}{\var{res} := \True}, Q) \equiv \{(C \land A) \lor (\neg C \land \neg A)\} \equiv \{C\Iff A\}$$

Para demostrar la correctitud de $\{Q_c\}S_{\text{if}}\{Q\}$ basta ver que $Q_c\implica \wp(S_{\text{if}}, Q) = \{C \Iff A\}$. Pero $Q_c$ implica que $\var{suma\_presidencial} = \predRef{suma}(\var{escrutinio\_presidencial})$, etc., y reemplazando estas igualdades en $C$ se obtiene $A$ y viceversa, con lo cual queda probado que $Q_c\implica \{C\Iff A\}$.

Para probar la correctitud parcial de $\{P_c\}S_c\{Q_c\}$, por el Teorema del Inviarante, basta demostrar que:
\begin{enumerate}
    \itemsep 0em
    \item $P_c\implica I$ 
    \item $\{I\land B\}S_c\{I\}$
    \item $I\land \neg B\implica Q_c$
\end{enumerate}
Demostremos cada ítem individualmente
\begin{enumerate}
    \item Notar que
    $$P_c\implica i = 0\implica 0\leq i\leq |\var{escrutinio\_presidencial}|$$
    pues $P_c\implica P$ y por $P$, vale $|\var{escrutinio\_presidencial}| > 0$. Por otro lado,
    $$P_c\implica i = 0\land \var{suma\_presidencial} = 0\implica \sum_{k=0}^{i-1}{\var{escrutinio\_presidencial}[i]} = 0$$
    dado que es la suma vacía.
    Estas dos cosas implican $I_1$. Ídem para $P_c \implica I_2$ y $P_c \implica I_3$. Además, $P_c \implica P$. Uniendo todo, $P_c \implica I$.

    \item Queremos ver que $\{I \land B\} \implica \wp(S_c, I)$, para lo cual calcularemos esta última wp.
    $$\wp(S_c, I) \equiv \wp(S1_c; S2_c; S3_c; S4_c, I)\equiv \wp(S1_c, \wp(S2_c, \wp(S3_c, \wp(S4_c, I)))$$
    donde $S1_c$, $S2_c$, $S3_c$ y $S4_c$ son las instrucciones dentro del ciclo.

    $$\wp(S4_c, I) \equiv \isDef(i)\yLuego I_{i+1}^{i} \equiv I_{i+1}^{i}\equiv
    Q_4$$
    $$\wp(S3_c, Q_4)\equiv \isDef(sd + ed[i]) \yLuego Q_4{}_{sd+ed[i]}^{sd} \equiv 0 \leq i < |ed| \yLuego Q_4{}_{sd+ed[i]}^{sd} \equiv Q_3$$
    $$\wp(S2_c, Q_3)\equiv \isDef(ss + es[i]) \yLuego Q_3{}_{ss+es[i]}^{ss} \equiv 0 \leq i < |es| \yLuego Q_3{}_{ss+es[i]}^{ss} \equiv Q_2$$ 
    $$\wp(S1_c, Q_2)\equiv \isDef(sp + ep[i]) \yLuego Q_2{}_{sp+ep[i]}^{sp} \equiv 0 \leq i < |ep| \yLuego Q_2{}_{sp+ep[i]}^{sp} \equiv Q_1$$ 
    donde $sd = \var{suma\_diputados}$, $ed = \var{escrutinio\_diputados}$, $ss = \var{suma\_senadores}$, $es = \var{escrutinio\_senadores}$, $sp = \var{suma\_presidencial}$ y $ep = \var{escrutinio\_presidencial}$. Es decir, la wp es equivalente a las tres condiciones de los rangos y a $I$ reemplazando $i$ por $i+1$ y agregando el nuevo término a cada sumatoria. Es decir:
    
    $$\wp(S_c, I)\equiv \{0 \leq i < |ep|, |ed|, |es| \yLuego |ep| = |ed| = |es| \land $$
    $$(0\leq i+1\leq |\var{ep}|\yLuego \var{sp+ep[i]} = \sum_{k=0}^{i}{\var{ep}[k]}) \land $$
    $$(0\leq i+1\leq |\var{ed}|\yLuego \var{sd+ed[i]} = \sum_{k=0}^{i}{\var{ed}[k]}) \land $$
    $$(0\leq i+1\leq |\var{es}|\yLuego \var{ss+es[i]} = \sum_{k=0}^{i}{\var{es}[k]}\}$$
    
    Notar que $0 \leq i < |ep|, |ed|, |es|\implica (0\leq i+1\leq |\var{ep}|) \land (0\leq i+1\leq |\var{ed}|) \land (0\leq i+1\leq |\var{es}|)$, por lo que podemos quitar estas condiciones de la wp, dado que son redundantes. Por otro lado, notar que
    $$sp + ep[i] = \sum_{k=0}^i{ep[k]}\iff sp = \sum_{k=0}^{i-1}{ep[k]}.$$

    Operando similarmente con las otras variables, podemos obtener que
    $$\wp(S_c, I)\equiv \{0 \leq i < |ep| = |ed| = |es| \yLuego \var{sp} = \sum_{k=0}^{i-1}{\var{ep}[k]} \land \var{sd} = \sum_{k=0}^{i-1}{\var{ed}[k]}) \land \var{ss} = \sum_{k=0}^{i-1}{\var{es}[k]}\}$$

    Solo queda ver que $\{I \land B\} \implica \wp(S_c, I)$. Sabemos que
    $$I\implica 0 \leq i\leq |ep| = |ed| = |es|\land \var{sp} = \sum_{k=0}^{i-1}{\var{ep}[k]}\land \var{sd} = \sum_{k=0}^{i-1}{\var{ed}[k]}\land \var{ss} = \sum_{k=0}^{i-1}{\var{es}[k]}$$
    Por otro lado, $B = \{i < |ep|\}$, entonces $I\land B$ implican $\{0 \leq i < |ep| = |ed| = |es|\}$. Queda demostrado, entonces, que $\{I \land B\} \implica \wp(S_c, I)$ y por lo tanto se cumple $\{I\land B\}S_c\{I\}$.
    
    \item Puede verse que $I \land \neg B \implica \{i = |ep| = |ed| = |es|\}$, lo que a su vez implica:
        \begin{itemize}
            \item $\var{sp} = \sum_{k=0}^{i-1}{\var{ep}[k]} = \predRef{suma}(\var{ep})$
            \item $\var{sd} = \sum_{k=0}^{i-1}{\var{ed}[k]} = \predRef{suma}(\var{ed})$
            \item $\var{ss} = \sum_{k=0}^{i-1}{\var{es}[k]} = \predRef{suma}(\var{es})$
        \end{itemize}
    Como $Q_c$ es la conjunción de estos tres predicados, queda claro que $I \land \neg B \implica Q_c$.
\end{enumerate}

Ahora, tenemos que probar la terminación del ciclo. Para eso, por el Teorema de Terminación de Ciclos, basta con demostrar:
\begin{enumerate}
    \item $\{I \land B \land f_v = v_0\}S_c\{f_v < v_0\}$
    \item $I \land (f_v \leq 0) \implica \neg B$
\end{enumerate}
Veamos que esto es cierto:
\begin{enumerate}
    \item Queremos ver que $\{I \land B \land f_v = v_0\} \implica \wp(S_c, f_v < v_0)$, para lo cual calcularemos esta última wp.
    $$\wp(S_c, f_v < v_0) \equiv \isDef(i+1)\land \wp(S1_c; S2_c; S3_c; S4_c, f_v < v_0)\equiv \wp(S1_c, \wp(S2_c, \wp(S3_c, \wp(S4_c, f_v < v_0)))$$
    En particular, podemos calcular
    $$\wp(S4_c, f_v < v_0))) \equiv \wp(i := i + 1, |ep| - i < v_0))) \equiv \{|ep| - i - 1 < v_0\} \equiv \{f_v < v_0 + 1\}$$
    Como las primeras tres líneas no alteran ninguna variable libre de la postcondición, sus wp son equivalentes a sus postcondiciones (las wp de las líneas siguientes). Por lo tanto, escribimos:
    $$\wp(S_c, f_v < v_0) \equiv 0\leq i < |ep|, |ed|, |es| \yLuego f_v < v_0 + 1$$
    dado que $\isDef(sp + ep[i])\Iff 0\leq i < |ep|$ y similarmente para las otras variables.
    
    Queda entonces ver que $\{I \land B \land f_v = v_0\} \implica \wp(S_c, f_v < v_0)$:
    \begin{itemize}
        \item $f_v = v_0$ implica $f_v < v_0 + 1$.
        \item $I \land B$ implica $0 \leq i < |ep|, |ed|, |es|$.
    \end{itemize}
    Entonces queda demostrado que $\{I \land B \land f_v = v_0\}\implica \wp(S_c, f_v < v_0)$ y por lo tanto se cumple la tripla de Hoare $\{I \land B \land f_v = v_0\}S_c\{f_v < v_0\}$.
    
    \item Dado que $f_v \leq 0 \Iff |ep| \leq i$, se ve que $I \land f_v \leq 0$ implica $\neg (i < |ep|) \equiv \neg B$.
\end{enumerate}

Habiendo demostrado la correctitud parcial y la terminación del ciclo, queda demostrada la correcitud total con la precondición de ciclo $P_c$, el invariante $I$ y la función variante $f_v$ propuestos.

Solo queda ver que $P \implica \wp(S1;S2;S3;S4, P_c)$, donde S1 a S4 son las primeras cuatro instrucciones del programa. Para ello, podemos calcular esta última wp. Como todas las opraciones en las instrucciones S1 a S4 están bien definidas,
$$\wp(S1;S2;S3;S4,P_c) \equiv \{0 = 0\land 0 = 0\land 0 = 0\land 0 = 0\land P\} \equiv \{\True\land P\} \equiv \{P\}$$
dado que reemplazamos todas las variables de $P_c$ por $0$.

Entonces queda demostrado que $P \implica \wp(S1;S2;S3;S4,P_c)$, pues $P \implica P$, y queda demostrada la correctitud del programa completo respecto a su especificación luego de aplicar el Corolario de la Monotonía, habiendo sido demostrada la correctitud de las distintas partes del programa (antes, en y después del ciclo).


\subsection{obtenerSenadoresEnProvincia}
Nombremos los predicados y auxiliares que vamos a usar en la demostración de correctitud del código de este programa:
\begin{itemize}
    \item $\predRef{sinRepetidos}(l) = \{\paraTodo[unalinea]{i, j}{\ent}{0\leq i < j < \longitud{l}\implicaLuego l[i]\neq l[j]}\}$
    \item $\predRef{sinNegativos}(l) = \{\paraTodo[unalinea]{i}{\ent}{0\leq i < \longitud{\var{l}}\implicaLuego \var{l}[i]\geq 0}\}$
    \item $\predRef{esIdDelMaximo}(l, i) = \{0 \leq i < \longitud{l} \yLuego
        \paraTodo[unalinea]{j}{\ent}{0\leq j < \longitud{l} \implicaLuego l[j]\leq l[i]}\}$
    \item $\predRef{esIdDelSegundoMaximo}(l, i) = \{0 \leq i < \longitud{l} \land
        \neg \predRef{esIdDelMaximo}(l, i) \land\\
        \paraTodo[unalinea]{j}{\ent}{(0\leq j < \longitud{l} \land \neg \predRef{esIdDelMaximo}(l, j))\implicaLuego l[j]\leq l[i]}\}$
    \item La precondición de la especificación: $P = \{\longitud{\var{escrutinio}} \geq 3 \land \predRef{sinRepetidos}(\var{escrutinio}) \land \predRef{sinNegativos}(\var{escrutinio})\}$
    \item La postcondición de la especificación: $Q = \{\predRef{esIdDelMaximo}(\predRef{subseq}(\var{escrutinio}, 0, \longitud{\var{escrutinio}} - 1), res_0)\land  \\ \predRef{esIdDelSegundoMaximo}(\predRef{subseq}(\var{escrutinio}, 0, \longitud{\var{escrutinio}} - 1), res_1)\}$
    \item La condición del primer if: $A = \{\var{escrutinio}[0] > \var{escrutinio}[1]\}$.
    \item La condición del ciclo: $B = \{i < |\var{escrutinio}| - 1\}$.
    \item La condición del primer if del ciclo: $C = \{\var{escrutinio}[i] > \var{escrutinio}[\var{indice\_maximo}]\}$.
    \item La condición del segundo if del ciclo: $D = \{\var{escrutinio}[i] > \var{escrutinio}[\var{indice\_segundo}]\}$.
\end{itemize}

Para demostrar la correctitud del ciclo, proponemos la siguiente pre y postcondición, invariante y función variante:
\begin{itemize}
    \item $P_c = \{P \yLuego i = 2 \land (A \land \var{indice\_maximo} = 0 \land \var{indice\_segundo} = 1) \lor (\neg A \land \var{indice\_maximo} = 1 \land\var{indice\_segundo} = 0)\}$.
    
    Por ser razonable que $P_c$ sea la conjunción entre $P$ y el estado resultante de las instrucciones previas al ciclo.
    
    \item $I = \{2 \leq i < |\var{escrutinio}| \land 0 \leq \var{indice\_maximo}, \var{indice\_segundo} < |\var{escrutinio}| - 1 \yLuego\\ \predRef{esIdDelMaximo}(\predRef{subseq}(\var{escrutinio}, 0, i), \var{indice\_maximo})\land \predRef{esIdDelSegundoMaximo}(\predRef{subseq}(\var{escrutinio}, 0, i), \var{indice\_segundo})\}$.
    
    Porque refleja lo que desde la intuición comprendemos que debe ocurrir al principio, durante (luego de cada iteración) y al final de la ejecución del ciclo.

    \item $Q_c = \{\predRef{esIdDelMaximo}(\predRef{subseq}(\var{escrutinio}, 0, \longitud{\var{escrutinio}} - 1), \var{indice\_maximo}) \land \\ \predRef{esIdDelSegundoMaximo}(\predRef{subseq}(\var{escrutinio}, 0, \longitud{\var{escrutinio}} - 1),  \var{indice\_segundo})\}$\\
    Porque esto implica que al finalizar el ciclo las variables $\var{indice\_maximo}$ e $\var{indice\_segundo}$ van a contener a lo que queremos luego guardar y devolver en la variable $\var{res}$.
    
    \item $f_v = |\var{escrutinio}| - 1 - i$\\
    Porque $f_v$ decrece en cada iteración del ciclo y se hace 0 luego de la última, cuando $i = |\var{escrutinio}| - 1$.
\end{itemize}

Comenzamos calculando la wp de la última instrucción del código:

$$\wp(\var{res} := (\var{indice\_maximo}, \var{indice\_segundo}), Q) \equiv Q_{\var{indice\_maximo}, \var{indice\_segundo}}^{\var{res}_0, \var{res}_1} \equiv$$
$$\{\predRef{esIdDelMaximo}(\predRef{subseq}(\var{escrutinio}, 0, \longitud{\var{escrutinio}} - 1), \var{indice\_maximo}) \land $$
$$\predRef{esIdDelSegundoMaximo}(\predRef{subseq}(\var{escrutinio}, 0, \longitud{\var{escrutinio}} - 1), \var{indice\_segundo})\} \equiv Q_c$$

Para ver que la instrucción que le sigue al ciclo es correcta queremos ver que
$$Q_c \implica \wp(\var{res} := (\var{indice\_maximo}, \var{indice\_segundo}), Q)$$
pero como vimos esa wp es equivalente a $Q_c$, y $Q_c \implica Q_c$ es tautológico. Entonces queda probada la correctitud de la porción del programa posterior al ciclo.

Para probar la correctitud parcial del ciclo, por el Teorema del Invariante basta con demostrar:
\begin{enumerate}
\itemsep 0em
    \item $P_c\implica I$ 
    \item $\{I\land B\}S_c\{I\}$
    \item $I\land \neg B\implica Q_c$
\end{enumerate}

Procedamos:

\begin{enumerate}
    \item Dado que $P_c$ implica que
    \begin{itemize}
        \item $|\var{escrutinio}|\geq 3 \land i = 2 \implica 2 \leq i < |\var{escrutinio}|$.
        \item $|\var{escrutinio}|\geq 3 \land \predRef{sinRepetidos}(\var{escrutinio})$ implica la excistencia de un primer y segundo máximo en la subsecuencia $\predRef{subseq}(\var{escrutinio}, 0, 2)$, según los predicados.
        \item $A\land \var{indice\_maximo} = 0 \land \var{indice\_segundo} = 1$ implica que
        $$\predRef{esIdDelMaximo}(\predRef{subseq}(\var{escrutinio}, 0, 2), 0) \land \predRef{esIdDelSegundoMaximo}(\predRef{subseq}(\var{escrutinio}, 0, 2), 1)$$
        y que los índices de los máximos están entre 0 y $|\var{escrutinio}|$.
        \item De forma análoga, si vale $\neg A \land \var{indice\_maximo} = 1 \land \var{indice\_segundo} = 0$, junto con la hipótesis de $\predRef{sinRepetidos}$, vale que
        $$\predRef{esIdDelMaximo}(\predRef{subseq}(\var{escrutinio}, 0, 2), 1) \land \predRef{esIdDelSegundoMaximo}(\predRef{subseq}(\var{escrutinio}, 0, 2), 0)$$
        y que los índices son válidos.
    \end{itemize}
    Entonces es cierto que $P_c \implica I$, como se quería probar.
    
    \item Queremos ver que $\{I \land B\} \implica \wp(S_c, I)$, para lo cual calcularemos esta última wp:
    $$\wp(i := i + 1, I) \equiv I_{i+1}^i$$
    $$\wp(S_c, I) \equiv \wp(\IfThenElse{C}{S_{11-12}}{S_{14-18}}, I_{i+1}^i) \equiv$$
    $$\{0 \leq i, \var{i\_max} < |\var{escrutinio}|\yLuego (C \land \wp(S_{11-12}, I_{i+1}^i)) \lor (\neg C \land \wp(S_{14-18}, I_{i+1}^i))) \}$$
    donde $S_{i-j}$ representa a la secuencia de instrucciones del programa de la línea $i$ a la línea $j$ inclusive, e $\var{i\_max} = \var{indice\_maximo}$ y $\var{i\_seg} = \var{indice\_segundo}$.

    Podemos entonces calcular la wp de cada parte por separado:
    $$\wp(S_{11-12}, I_{i+1}^i) \equiv \wp(\var{i\_seg} := \var{i\_max}, \wp(\var{i\_max} := i, I_{i+1}^i)) \equiv I_{i+1, \var{i\_max}, i}^{i, \var{i\_seg}, \var{i\_max}}$$
    

    $$\wp(S_{14-18}, I_{i+1}^i) \equiv \wp(\IfThenElse{D}{\var{i\_seg} := i}{skip}) \equiv$$
    $$\{0 \leq i, \var{i\_seg} < |\var{escrutinio}| \yLuego ((D \land \wp(\var{i\_seg} := i, I_{i+1}^i)) \lor (\neg D \land \wp(skip, I_{i+1}^i)))\} \equiv$$
    $$\{0 \leq i, \var{i\_seg} < |\var{escrutinio}| \yLuego ((D \land I_{i+1,i}^{i, \var{i\_seg}}) \lor (\neg D \land I_{i+1}^i))\}$$
    
    Entonces,
    $$\wp(S_c, I) \equiv 0 \leq i, i\_max < |esc| \yLuego (C \land I_{i+1, \var{i\_max}, i}^{i, \var{i\_seg}, \var{i\_max}}) \lor (\neg C \land (0 \leq i, i\_seg < |esc| \yLuego ((D \land I_{i+1,i}^{i,i\_seg}) \lor (\neg D \land I_{i+1}^i)))))$$

    Desglosando esta última igualdad:    
    \begin{itemize}
        \item $I_{i+1}^{i} \equiv \{2 \leq i + 1 < |\var{esc}| \land 0 \leq i\_max, i\_seg < |\var{esc}| - 1 \yLuego\\ \predRef{esIdDelMaximo}(\predRef{subseq}(\var{esc}, 0, i + 1), i\_max) \land\\ \predRef{esIdDelSegundoMaximo}(\predRef{subseq}(\var{esc}, 0, i + 1), i\_seg)\}$
        \item $I_{i+1,i}^{i,i\_seg} \equiv \{2 \leq i + 1 < |\var{esc}| \land 0 \leq i\_max, i < |\var{esc}| - 1 \yLuego\\ \predRef{esIdDelMaximo}(\predRef{subseq}(\var{esc}, 0, i + 1), i\_max) \land\\ \predRef{esIdDelSegundoMaximo}(\predRef{subseq}(\var{esc}, 0, i + 1), i)\}$
        \item $I_{i+1, \var{i\_max}, i}^{i, \var{i\_seg}, \var{i\_max}} \equiv \{2 \leq i + 1 < |\var{esc}| \land 0 \leq i, i\_max < |\var{esc}| - 1 \yLuego\\ \predRef{esIdDelMaximo}(\predRef{subseq}(\var{esc}, 0, i + 1), i) \land\\ \predRef{esIdDelSegundoMaximo}(\predRef{subseq}(\var{esc}, 0, i + 1), i\_max)\}$
    \end{itemize}

    Notemos que parte del invariante y $B$, $(2 \leq i < |\var{esc}| \land 0 \leq i\_max, i\_seg < |\var{esc}| - 1 \land i < |\var{esc}| - 1)$, implica $(0 \leq i, i\_max, i\_seg < |esc| - 1) \land (2 \leq i + 1 < |esc|)$, que a su vez implica $(0 \leq i, i\_max, i\_seg < |esc|)$. Para ver que $\{I \land B\} \implica \wp(S_c, I)$, basta con ver que esto se cumple al analizar los tres casos posibles dentro del ciclo (o más bien, sus paralelos dentro de la \wp):
    \begin{itemize}
        \item Caso $C \equiv esc[i] > esc[i\_max].$ En este caso, para que $\{I \land B\} \implica \wp(S_c, I)$ basta con que $\{I \land B\} \implica \{0 \leq i, i\_max < |esc| \yLuego C \land I_{i+1, \var{i\_max}, i}^{i, \var{i\_seg}, \var{i\_max}}\}$ (ya que es uno de los terminos separados por una disyuncion logica del resto de los terminos). Como ya comentamos, $\{I \land B\} \implica \{0 \leq i, i\_max < |esc|\}$, y estamos asumiendo que estamos en el caso en que vale $C$. Entonces queda ver que, asumiendo $C$, $\{I \land B\} \implica \{I_{i+1, \var{i\_max}, i}^{i, \var{i\_seg}, \var{i\_max}}\}$. Esto es verdadero ya que $I \land B \land C$ implica que $i$ es el índice del máximo de la subsecuencia $subseq(esc, 0, i+1)$ y que por lo tanto $i$ es el nuevo $i\_max$ y el $i\_max$ anterior es el nuevo $i\_seg$, que es precisamente lo que está expresado en $I_{i+1, \var{i\_max}, i}^{i, \var{i\_seg}, \var{i\_max}}$. Además, como ya se comentó, $I \land B \implica 2 \leq i + 1 < |\var{esc}| \land 0 \leq i, i\_max < |\var{esc}| - 1$. Entonces vale que $\{I \land B\} \implica \{I_{i+1, \var{i\_max}, i}^{i, \var{i\_seg}, \var{i\_max}}\}$ y por lo tanto también que $\{I \land B\} \implica \wp(S_c, I)$ si vale $C$.

        \item Caso $\neg C \land D \equiv esc[i] \leq esc[i\_max] \land esc[i] > esc[i\_seg]$. Para que $\{I \land B\} \implica \wp(S_c, I)$ basta con que $\{I \land B\} \implica \{0 \leq i, i\_max < |esc| \yLuego \neg C \land (0 \leq i, i\_seg < |esc| \yLuego D \land I_{i+1,i}^{i,i\_seg})\}$. Ya comentamos que $\{I \land B\} \implica \{0 \leq i, i\_max, i\_seg < |esc|\}$, y estamos asumiendo que estamos en el caso en que vale $\neg C \land D$. Queda ver que, asumiendo $\neg C \land D$, $\{I \land B\} \implica \{I_{i+1,i}^{i,i\_seg}\}$. Esto es verdadero pues $I \land \land B \neg C \land D$ implica que i es el índice del segundo máximo de la subsecuencia $subseq(esc, 0, i+1)$ y que por lo tanto $i$ es el nuevo $i\_seg$ y el $i\_max$ anterior no es afectado, exáctamente lo que expresa $I^{i+1,i}_{i,i\_seg}$. Además, como ya se comentó, $I \land B \implica 2 \leq i + 1 < |\var{esc}| \land 0 \leq i\_max, i < |\var{esc}| - 1$. Entonces vale que $\{I \land B\} \implica \{I_{i+1, i}^{i, i\_seg}\}$ y por lo tanto también que $\{I \land B\} \implica \wp(S_c, I)$ si vale $\neg C \land D$.
    
        \item Caso $\neg C \land \neg D \equiv esc[i] \leq esc[i\_max] \land esc[i] \leq esc[i\_seg]$. Para que $\{I \land B\} \implica \wp(S_c, I)$ basta con que $\{I \land B\} \implica \{0 \leq i, i\_max < |esc| \yLuego \neg C \land (0 \leq i, i\_seg < |esc| \yLuego \neg D \land I_{i+1}^{i})\}$. Ya comentamos que $\{I \land B\} \implica \{0 \leq i, i\_max, i\_seg < |esc|\}$, y estamos asumiendo que estamos en el caso en que vale $\neg C \land \neg D$. Queda ver que, asumiendo $\neg C \land \neg D$, $\{I \land B\} \implica \{I_{i+1}^{i}\}$. Esto es verdadero considerando que $I \land B \land \neg C \land \neg D$ implica que no se tiene ni al máximo ni al segundo máximo de $subseq(esc, 0, i+1)$ en la posición $i$ y que por lo tanto $i\_max$ e $i\_seg$ no son modificados en la iteración, que es exáctamente lo que expresa $I_{i+1}^{i}$. Por último, sabemos que $I \land B \implica 2 \leq i + 1 < |\var{esc}| \land 0 \leq i\_max, i\_seg < |\var{esc}| - 1$. Entonces vale que $\{I \land B\} \implica \{I_{i+1}^{i}\}$ y por lo tanto también que $\{I \land B\} \implica \wp(S_c, I)$ si vale $\neg C \land \neg D$.
    \end{itemize}

    Dado que siempre se cumple que $(C \lor (\neg C \land (D \lor \neg D))) \equiv ((C) \lor (\neg C \land D) \lor (\neg C \land \neg D))$ (es decir, ocurre uno de estos 3 casos analizados) y a que se probó que en todos los casos vale $\{I \land B\} \implica \wp(S_c, I)$, entonces es verdad que $\{I \land B\} \implica \wp(S_c, I)$ en general, y por lo tanto tambíen que $\{I\land B\}S_c\{I\}$ como se quería demostrar.
    \item $I \land \neg B \implica i = |\var{escrutinio}| - 1$, condición bajo la cual $Q_c$ es idéntico a parte de $I$. Por lo tanto es inmediato que $I\land \neg B\implica Q_c$.
\end{enumerate}
Para probar la terminación del ciclo, debemos demostrar:
\begin{enumerate}
    \item $\{I \land B \land f_v = v_0\}S_c\{f_v < v_0\}$
    \item $I \land f_v \leq 0 \implica \neg B$
\end{enumerate}
Veamos que esto es cierto:
\begin{enumerate}
    \item Queremos ver que $\{I \land B \land f_v = v_0\} \implica \wp(S_c, f_v < v_0)$, para lo cual calcularemos esta última wp.
    
    $\wp(i := i + 1, |esc| - 1 - i < v_0) \equiv \{|esc| - 1 - i - 1 < v_0\} \equiv \{f_v < v_0 + 1\}$
    \\\\
    $\wp(S_c, f_v < v_0) \equiv \wp(\IfThenElse{C}{S_{11-12}}{S_{14-18}}, f_v < v_0 + 1)$\\
    Como $f_v < v_0 + 1$ no depende de $i\_max$ ni de $i\_seg$, esto resulta en:
    
    $\wp(S_c, f_v < v0) \equiv 0 \leq i, i\_max < |esc| \yLuego ((C \land f_v < v_0+1) \lor (\neg C \land (0 \leq i, i\_seg < |esc| \yLuego ((D \land f_v < v_0 + 1) \lor (\neg D \land f_v < v_0 + 1)))))$

    Luego, dado que
    \begin{itemize}
        \item $f_v = v_0 \implica f_v < v_0+1$
        \item $I \land B \implica 0 \leq i, i\_max, i\_seg < |esc|$
        \item Siempre se cumple $(C \lor (\neg C \land (D \lor \neg D)))$
    \end{itemize}
    Se vé que $\{I \land B \land f_v = v_0\} \implica \wp(S_c, f_v < v_0)$ y que por lo tanto queda probado que $\{I \land B \land f_v = v_0\}S_c\{f_v < v_0\}$.
    \item Considerando que $f_v \leq 0 \Iff i \geq |esc| - 1$ se vé que $I \land f_v \leq 0 \implica i = |esc| - 1$, condición bajo la cual vale $\neg B$. Por lo tanto, $I \land f_v \leq 0 \implica \neg B$.
\end{enumerate}

Demostrada la correctitud parcial y la terminación del ciclo, queda entonces demostrada su correctitud total. 

Para ver que el programa entero es correcto respecto de su especificación, queda ver que $P \implica wp(S_{1-8}, P_c)$

$wp(i := 2, P_c) \equiv \{P \yLuego 2 = 2 \land ((esc[0] > esc[1] \land i\_max = 0 \land i\_seg = 1) \lor (esc[1] > esc[0] \land i\_max = 1 \land i\_seg = 0))\} \equiv \\ \{P \yLuego ((esc[0] > esc[1] \land i\_max = 0 \land i\_seg = 1) \lor (esc[1] > esc[0] \land i\_max = 1 \land i\_seg = 0))\} \equiv Q_8$\\\\
$wp(S_{1-8}, P_c) \equiv wp(\IfThenElse{esc[0]>esc[1]}{i\_max := 0; i\_seg := 1}{i\_max := 1; i\_seg := 0}, Q_8) \equiv \\ 1 < |esc| \yLuego ((esc[0]>esc[1] \land wp(i\_max := 0, wp(i\_seg := 1,Q_8))) \lor (esc[0] \leq esc[1] \land wp(i\_max := 1, wp(i\_seg := 0,Q_8))))$\\\\
$wp(i\_max := 0, wp(i\_seg := 1,Q_8)) \equiv \{P \yLuego ((esc[0] > esc[1] \land 0 = 0 \land 1 = 1) \lor (esc[1] > esc[0] \land 0 = 1 \land 1 = 0))\} \equiv \{P \yLuego esc[0] > esc[1]\}$\\\\
$wp(i\_max := 1, wp(i\_seg := 0,Q_8)) \equiv \{P \yLuego ((esc[0] > esc[1] \land 1 = 0 \land 0 = 1) \lor (esc[1] > esc[0] \land 1 = 1 \land 0 = 0))\} \equiv \{P \yLuego esc[0] \leq esc[1]\}$\\\\
$wp(S_{1-8}, P_c) \equiv \{1 < |esc| \yLuego P\}$

Dado que
\begin{itemize}
    \item $P \implica P$
    \item $P \implica 1 < |esc|$
\end{itemize}

Se vé que $P \implica wp(S_{1-8}, P_c)$, con lo que queda demostrada la correctitud del programa previo al ciclo. Demostrada la correctitud del programa antes, durante y después del ciclo, por monotonía queda demostrada la correctitud del programa completo.

% Lo principal: las fórmulas. Se puede poner en una linea, como $x_i = x_{i-1} + x_{i-2}$, o ponerse más grande:

% \begin{equation}
% 	\sum\limits_{i=0}^{n} i
% 	\label{eq:1}
% \end{equation}

% Y se pueden citar ecuaciones con \verb|\eqref{nombreDeEq}|: \eqref{eq:1}

% Ejemplo de itemizado:

% \begin{itemize}
% 	\item Item 1
% 	\item Item 2
% 	\item Item 3
% \end{itemize}

% Ejemplo de enumerado con menor distancia entre items:

% \begin{enumerate} \setlength\itemsep{0cm}
% 	\item Item 1
% 	\item Item 2
% 	\item Item 3
% \end{enumerate}

% Podemos escribir mucho texto. Mucho texto. Mucho texto. Mucho texto. Mucho texto. Mucho texto. Mucho texto. Mucho texto. Mucho texto. Mucho texto. Mucho texto.

% Otro párrafo. Otro párrafo. Otro párrafo. Otro párrafo. Otro párrafo. Otro párrafo. Otro párrafo. Otro párrafo. Otro párrafo. Otro párrafo. Otro párrafo. Otro párrafo. Otro párrafo.

% \vspace{0.3cm}

% Le agregamos una separación entre párrafos. Le agregamos una separación entre párrafos. Le agregamos una separación entre párrafos. Le agregamos una separación entre párrafos. Le agregamos una separación entre párrafos.

% \vspace{0.3cm}
% La tabla \ref{tab:ejemplo} es un ejemplo de cómo se hace una tabla.

% \begin{table}[h!]
% 	\centering
% 	\begin{tabular}{||l c c r||} 
% 		\hline
% 		Col1 & Col2 & Col2 & Col3 \\ [0.5ex] 
% 		\hline\hline
% 		1 & 6 & 87837 & 787 \\ 
% 		2 & 7 & 78 & 5415 \\
% 		3 & 545 & 778 & 7507 \\
% 		4 & 545 & 18744 & 7560 \\
% 		5 & 88 & 788 & 6344 \\
% 		\hline
% 	\end{tabular}
% 	\caption{Ejemplo de tabla}
% 	\label{tab:ejemplo}
% \end{table}


% % Para hacer que quede todo en una misma linea, se puede usar minipage
% %\begin{minipage}[t]{\textwidth}
% 	\begin{lstlisting}[caption={Ejemplo de código (usando los estilos de la cátedra, ver las macros para más detalles)},label=code:for]
% res := 0;
% i := 0;
% while (i < s.size()) do
% 	res := res + s[i];
% 	i := i + 1
% endwhile
% 	\end{lstlisting}
% %\end{minipage}

% Si se pone un label al \verb|lstlisting|, se puede referenciar: Código \ref{code:for}.


% \subsection{Macros de la cátedra para especificar}

% \begin{proc}{nombre}{\In paramIn : \nat, \Inout paramInout : \TLista{\ent}}{tipoRes}
% 	%    \modifica{parametro1, parametro2,..}
% 	\requiere{expresionBooleana1}
% 	\asegura{expresionBooleana2}
% 	\aux{auxiliar1}{parametros}{tipoRes}{expresion}
% 	\pred{pred1}{parametros}{expresion} 
% \end{proc}

% \aux{auxiliarSuelto}{parametros}{tipoRes}{expresion}
% % \paraTodo{variable}{tipo}{expresion}
% % \existe{variable}{tipo}{expresion}
% % Pueden tener [unalinea] para que no se divida en varias lineas
% \pred{predSuelto}{parametros}{\paraTodo[unalinea]{variable}{tipo}{algo \implicaLuego expresion}}
% \pred{predSuelto}{parametros}{\existe[unalinea]{variable}{tipo}{algo \yLuego expresion}}

\end{document}

Integrante LU Correo electr ́onico
Apellido, Nombre1 001/01 email1@dominio.com
Apellido, Nombre2 002/01 email2@dominio.com
Apellido, Nombre3 003/01 email3@dominio.com
Apellido, Nombre4 004/01 email4@dominio.com